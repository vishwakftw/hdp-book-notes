\documentclass{article}
\usepackage[margin=1in]{geometry}
\usepackage{amsmath, amssymb, amsthm}
\usepackage{hyperref}
\usepackage{nicefrac}

\setlength{\parskip}{2mm}

\newtheorem{theorem}{Theorem}[section]
\newtheorem{corollary}{Corollary}[section]
\newtheorem{lemma}{Lemma}[section]

\theoremstyle{remark}
\newtheorem*{remark}{Remark}

\theoremstyle{defintiion}
\newtheorem{definition}{Definition}

\newcommand{\real}{\mathbb{R}}
\newcommand{\Exp}{\mathbb{E}}
\newcommand{\Var}{\mathrm{Var}}
\newcommand{\Cov}{\mathrm{Cov}}
\newcommand{\inner}[2]{\left\langle #1, #2 \right\rangle}
\newcommand{\indic}[1]{\mathbf{1}_{\{#1\}}}
\newcommand{\trace}{\mathrm{trace}}
\newcommand{\OPT}{\mathrm{OPT}}

\newcommand{\eparam}{\widehat{\theta}}
\newcommand{\tparam}{\theta^{\star}}

\title{Fast Mean Estimation with Sub-Gaussian Rates}
\author{Cherapanamjeri, Flammarion, Bartlett}

\date{}

\begin{document}
\maketitle

\raggedright

\section{Introduction}
\subsection{Goal}
To obtain high probability mean estimates when only the existence of the \(2^{nd}\) moment is known. This is also called the \emph{heavy tailed} setting, where higher order moments from the sampling distribution need not exist.

\subsection{Existing results}
Consider the estimator to be the sample mean \(\eparam = \frac{1}{n}\sum\limits_{i=1}^{n}X_{i}\) where \(\{X_{i}\}_{i=1}^{n}\) are sampled from a distribution \(P\) with only finite \(2^{nd}\) moment and mean \(\tparam\). Markov's inequality gives:
\begin{equation*}
\Pr(\|\eparam - \tparam\|_{2} > t) \leq \frac{\Exp[\|\eparam - \tparam\|^{2}_{2}]}{t^{2}}
\end{equation*}

Note that \(\eparam - \tparam = \frac{1}{n}\sum\limits_{i=1}^{n}(X_{i} - \tparam)\) and hence:
\begin{equation*}
\|\eparam - \tparam\|_{2}^{2} = \frac{1}{n^{2}}\sum_{i=1}^{n} \|X_{i} - \tparam\|_{2}^{2} + \frac{1}{n}\sum_{\substack{i, j = 1 \\ i \neq j}}^{n} (X_{i} - \tparam)^{T}(X_{j} - \tparam) \Rightarrow \Exp[\|\eparam - \tparam\|_{2}^{2}] = \frac{1}{n^{2}}\sum_{i=1}^{n} \Exp\left[\|X_{i} - \tparam\|_{2}^{2}\right]
\end{equation*}

and since \(\Exp\left[\|X_{i} - \tparam\|_{2}^{2}\right] = \Exp\left[\trace(X_{i} - \tparam)(X_{i} - \tparam)^{T}\right] = \Sigma\), we get:
\begin{equation*}
\Exp[\|\eparam - \tparam\|_{2}^{2}] = \frac{\trace(\Sigma)}{n}
\end{equation*}
therefore leading to:
\begin{equation*}
\Pr\left(\|\eparam - \tparam\|_{2} > \sqrt{\frac{\trace(\Sigma)}{n\delta}}\right) \leq \delta
\end{equation*}
which corresponds to: with probability at least \(1 - \delta\):
\begin{equation*}
\|\eparam - \tparam\|_{2} \leq \sqrt{\frac{\trace(\Sigma)}{n\delta}}
\end{equation*}

In contrast, when \(P\) is Gaussian, we get:
\begin{equation*}
\Pr\left(\|\eparam - \tparam\|_{2} > O\left(\sqrt{\frac{\trace(\Sigma)}{n}} + \sqrt{\frac{\|\Sigma\|_{2}\log(\nicefrac{1}{\delta})}{n}}\right)\right) \leq \delta
\end{equation*}

We will denote \(\sqrt{\frac{\trace(\Sigma)}{n}} + \sqrt{\frac{\|\Sigma\|_{2}\log(\nicefrac{1}{\delta})}{n}}\) as \(\OPT_{n, \delta, \Sigma}\) as a shorthand.

To show this, consider \(Z_{i} = X_{i} - \tparam\) for all \(i \in [n]\). Then \(\|\eparam - \tparam\|_{2} = \left\|\frac{1}{n}\sum\limits_{i=1}^{n}Z_{i}\right\|_{2}\), where \(Z_{i}\)s are zero mean Gaussian RVs with covariance \(\Sigma\). Note that \(Z_{i} = \Sigma^{\nicefrac{1}{2}}Y_{i}\) for all \(i \in [n]\) where \(Y_{i}\)s are standard multivariate Gaussian RVs. Now, we have that:
\begin{equation*}
|\|Z_{i}\| - \|Z'_{i}\|_{2}| \leq \|Z_{i} - Z'_{i}\|_{2} \leq \|\Sigma^{\nicefrac{1}{2}}(Y_{i} - Y'_{i})\|_{2} \leq \|\Sigma^{\nicefrac{1}{2}}\|_{2}\|Y_{i} - Y'_{i}\|_{2}
\end{equation*}

which shows that \(\|Z_{i}\|\) is a \(\|\Sigma^{\nicefrac{1}{2}}\|_{2}\)-Lipschitz function of \(Y_{i}\). By a Lipschitz concentration lemma due to Tsirelson, Ibragimov and Sudakov, we have:
\begin{equation*}
\Pr\left(\left\|\frac{1}{n} \sum_{i=1}^{n} Z_{i}\right\|_{2} - \Exp\left[\left\|\frac{1}{n} \sum_{i=1}^{n} Z_{i}\right\|_{2}\right] > t\right) \leq \exp\left(-\frac{nt^{2}}{2\|\Sigma\|_{2}}\right)
\end{equation*}
leading to:
\begin{equation*}
\Pr\left(\left\|\frac{1}{n} \sum_{i=1}^{n} Z_{i}\right\|_{2} > \Exp\left[\left\|\frac{1}{n} \sum_{i=1}^{n} Z_{i}\right\|_{2}\right] + t\right) \leq \exp\left(-\frac{nt^{2}}{2\|\Sigma\|_{2}}\right)
\end{equation*}

and with probability at least \(1 - \delta\):
\begin{multline*}
\|\eparam - \tparam\|_{2} \leq \Exp[\|\eparam - \tparam\|_{2}] + \sqrt{\frac{2\|\Sigma\|_{2}\log(\nicefrac{1}{\delta})}{n}} \leq \sqrt{\Exp[\|\eparam - \tparam\|_{2}^{2}]} + \sqrt{\frac{2\|\Sigma\|_{2}\log(\nicefrac{1}{\delta})}{n}} \\\leq \sqrt{\frac{\trace(\Sigma)}{n}} + \sqrt{\frac{2\|\Sigma\|_{2}\log(\nicefrac{1}{\delta})}{n}}
\end{multline*}

Lugosi and Mendelson showed that with only bounded \(2^{nd}\) moment, this rate can be achieved, but the estimator proposed is intractable.

\section{Main Result}
\begin{theorem}
Let \(\{X_{i}\}_{i=1}^{n}\) be a set of \(n\) i.i.d. random vectors i.e. \(X_{i} \in \real^{p}\), sampled from a distribution with mean \(\tparam\) and covariance \(\Sigma\). Then \(\mathsf{Descent-Mean-Estimate}\) with stepsize \(\gamma = \frac{1}{20}\) and number of iterations \(T = 1000\frac{\log(\|\tparam\|_{2})}{\epsilon}\) returns a mean estimate \(\eparam_{n, \delta}\) that satisfies with probability at least \(1 - \delta\):
\begin{equation*}
\|\eparam_{n, \delta} - \tparam\|_{2} \leq \max\left(\epsilon, 480000 \cdot \OPT_{n, \delta, \Sigma}\right)
\end{equation*}
\end{theorem}

\(\mathsf{Descent-Mean-Estimate}\) is Algorithm 1 in the main text.



\end{document}
